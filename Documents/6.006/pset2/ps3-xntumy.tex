%
% 6.006 homework template
%
% Please rename this document to ps1-jdoe.tex (where jdoe is your
% Athena username) and turn in ps1-jdoe.pdf.
%
% You can compile this document to PDF by typing `pdflatex ps1-jdoe.tex`.
%
% You will need a font included in the texlive-fonts-recommended
% package (which can be installed via apt, or your package manager of
% choice) or the document won't build and you'll get an error about a
% missing font.
%
% NOTE:
% Be sure to define your name with the \name command
% Be sure to use the \answer command for each of your answers 
%   (first argument: problem name
%   second argument: collaborators (write 'none' if you solved it alone))
\documentclass[12pt]{article}

\newcommand{\yourname}{Xola Ntumy}
\newcommand{\problemset}{ Problem set 3 }

\input{macros}

%\pagestyle{headings}
\usepackage{graphicx}
\usepackage{amsfonts}
\usepackage{amssymb}
\usepackage{amsmath}
\usepackage{latexsym}
\usepackage{enumerate}
\usepackage{mdwlist}

\setlength{\parskip}{1pc}
\setlength{\parindent}{0pt}
\setlength{\topmargin}{-3pc}
\setlength{\textheight}{9.5in}
\setlength{\oddsidemargin}{0pc}
\setlength{\evensidemargin}{0pc}
\setlength{\textwidth}{6.5in}

\newcommand{\theproblemsetnum}{1}
\newcommand{\releasedate}{March 1, 2011}
\newcommand{\partaduedate}{Monday, March 7}
\newcommand{\tabUnit}{3ex}
\newcommand{\tabT}{\hspace*{\tabUnit}}


\newcommand{\answer}[2]{
\newpage
\noindent
\framebox{
	\vbox{
		6.006 Homework \hfill {\bf \problemset}
		\hfill \# #1 \\ 
		\yourname \hfill \today \\
                Collaborators: #2
	}
}
\bigskip

}


\begin{document}


\answer{1 -- Recurrences}{Kwadwo Nyarko} 
%Write `none' if solved alone.

\begin{enumerate}[(a)]

\item
T(n) = O(n log n)

\end{enumerate}


\begin{enumerate}[(b)]

\item
T(n) =   O (n to the power log base 4 of 3)


\end{enumerate}

\begin{enumerate}[(c)]

\item
T(n) = O(n to the power 2)

\end{enumerate}


\begin{enumerate}[(d)]

\item
The recurrence relation would be: T(n) = c * T(n / c) + log n

Solving this relation would give us: 

T(n) = O (n log n)

\end{enumerate}


\begin{enumerate}[(e)]

\item
The running time ceases to be constant. It would now change based on the function n.

\end{enumerate}

% PROBLEM 2

\answer{2 -- $d$-ary Heaps}{Kwadwo Nyarko}
%Write `none' if solved alone.

\begin{enumerate}[(a)]

\item
Parent(i):

return ((i - 1) / d ) + 1

\end{enumerate}

\begin{enumerate}[(b)]

\item
Child(i, k):

return (i * d) + k + 1

\end{enumerate}


\begin{enumerate}[(c)]

\item
The height of the d-array heap would be O(log base d of n)

\end{enumerate}

\begin{enumerate}[(d)]

\item
The asymptotic running time of Heapify is O(d * log base d of n) since for every level it does d comparisons and it is of height log base d of n.

while the running time of Increase-key will be O(log base d of n) since it does one comparison per each of the log base d of n levels in the heap.

\end{enumerate}

\begin{enumerate}[(e)]

\item
Since the height is O(log base d of n):
 
If d = O(1), the resulting running time is O(log n)

If d = O(log n), the resulting running time becomes O(log base log n of n) which simplifies to O(log n / (log (log n)))

If d = O(n), the resulting running time becomes O(log base n of n) which simplifies to O(1)

\end{enumerate}

\begin{enumerate}[(f)]

\item
The running times for Heapify:

a) when d = O(1), would be O(log n)

b) when d = O(log n), would be O(log n * log n / (log (log n)))

c) when d = O(n), would be O(n)

The running times for Increase-key:

a) when d = O(1), would be O(log n)

b) when d = O(log n), would be O(log n / (log (log n)))

c) when d = O(n), would be O(1)

When d is increased the running times increase for Heapify but decrease for Increase-Key

\end{enumerate}


\end{document}

