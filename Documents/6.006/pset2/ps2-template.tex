%
% 6.006 homework template
%
% Please rename this document to ps1-jdoe.tex (where jdoe is your
% Athena username) and turn in ps1-jdoe.pdf.
%
% You can compile this document to PDF by typing `pdflatex ps1-jdoe.tex`.
%
% You will need a font included in the texlive-fonts-recommended
% package (which can be installed via apt, or your package manager of
% choice) or the document won't build and you'll get an error about a
% missing font.
%
% NOTE:
% Be sure to define your name with the \name command
% Be sure to use the \answer command for each of your answers 
%   (first argument: problem name
%   second argument: collaborators (write 'none' if you solved it alone))
\documentclass[12pt]{article}

\newcommand{\yourname}{INSERT NAME HERE}
\newcommand{\problemset}{ Problem set 2 }

\input{macros}

%\pagestyle{headings}
\usepackage[dvips]{graphics,color}
\usepackage{amsfonts}
\usepackage{amssymb}
\usepackage{amsmath}
\usepackage{latexsym}
\usepackage{enumerate}
\usepackage{mdwlist}

\setlength{\parskip}{1pc}
\setlength{\parindent}{0pt}
\setlength{\topmargin}{-3pc}
\setlength{\textheight}{9.5in}
\setlength{\oddsidemargin}{0pc}
\setlength{\evensidemargin}{0pc}
\setlength{\textwidth}{6.5in}

\newcommand{\theproblemsetnum}{1}
\newcommand{\releasedate}{February 15, 2011}
\newcommand{\partaduedate}{Monday, February 28}
\newcommand{\tabUnit}{3ex}
\newcommand{\tabT}{\hspace*{\tabUnit}}


\newcommand{\answer}[2]{
\newpage
\noindent
\framebox{
	\vbox{
		6.006 Homework \hfill {\bf \problemset}
		\hfill \# #1 \\ 
		\yourname \hfill \today \\
                Collaborators: #2
	}
}
\bigskip

}


\begin{document}


\answer{1 --  Hash functions and load}{INSERT COLLABORATORS HERE} 
%Write `none' if solved alone.

\begin{enumerate}[(a)]

\item
The algorithm will not have the same performance as the one in lecture. Provided the size of her hash table is prime number, the hash function will assign keys randomly. However, there will be the possibility of collisions because some phrases may be repeated and some words may have the same characters (letters). Also words containing the same characters but with different permutations would hash to the same value causing more collisions.
The solution would be either to result to chaining or to use a more efficient hash function. For instance a hash function where each letter is multiplied by a base raised to a power depending on its position so that different permutations would result in different hash values.


\end{enumerate}

\begin{enumerate}[(b)]

\item
This algorithm will not have the same performance as the one in lecture. The reason for this is that the size of the table is a power of two, 2^p, this makes h(k) the p lowest-order bits of k. Permuting the characters of k will not change the value of h(k).
Also the items are ordered...

\end{enumerate}

\begin{enumerate}[(c)]

\item


\end(enumerate}



% PROBLEM 2

\answer{2 -- Collision resolution and dynamic resizing}{INSERT COLLABORATORS HERE}
%Write `none' if solved alone.

\begin{enumerate}[(a)]

\item
We must consider dynamically resizing our table because there will be an arbitrary sequence of inserts and deletes and there is the possibility that the table will become full. Therefore there must be a strategy to implement in the case where the table is full.

\end{enumerate}


\begin{enumerate}[(b)]

\item
On the other hand, resizing whenever we have a collision will be much too expensive since this will require rehashing all the slots in the table. We would like the time for an insert to be O(1) but it will be much worse.

\end{enumerate}

% PROBLEM 3

\answer{3 -- Python dictionaries}

\begin{enumerate}[(a)]

\item
Examples of use cases are:
-Passing keyword arguments
-Class method lookup
-Instance attribute lookup and global variables
-Builtins
-Uniquification
-Membership testing
-Dynamic mappings

\end{enumerate}

\begin{enumerate}[(b)]

\item
This use case will deploy one hash table that contains the details of the given members. It will be created once and need not be altered significantly after its creation. Calls will often by made to has_key() or __contains__() to search out and test the validity of a supposed member.

\end{enumerate}

% PROBLEM 4

\answer{4 -- Matching DNA sequences}{INSERT COLLABORATORS HERE}
%Write `none' if solved alone.

\begin{enumerate}[(a)]

\item

\end{enumerate}

\end{document}

